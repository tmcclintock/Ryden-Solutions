\documentclass[12pt,a4paper]{article}
\usepackage{amsmath}
\usepackage[english]{babel}
\usepackage{graphicx}
\usepackage{listings}
\usepackage{fullpage}
\usepackage[T1]{fontenc}
\usepackage{enumerate}
\usepackage[makeroom]{cancel}
\usepackage{hyperref}
\usepackage{natbib}
\bibliographystyle{apj}
\usepackage[labelfont=bf]{caption}
\usepackage[labelsep=period]{caption}

\lstdefinestyle{custompython}{
  belowcaptionskip=1\baselineskip,
  breaklines=true,
  frame=L,
  xleftmargin=\parindent,
  language=bash,
  basicstyle=\footnotesize\ttfamily,
  showstringspaces=false,
  %commentstyle=\itshape\color{purple!40!black},
  %keywordstyle=\itshape\color{green!40!black},
  %identifierstyle=\color{blue},
  %stringstyle=\color{orange},
}

\usepackage{todonotes}
\newcommand\TM[1]{\todo[color=green!40, inline, size=\small]{TM: #1}}

% Roman numerals command
\makeatletter
\newcommand*{\rom}[1]{\expandafter\@slowromancap\romannumeral #1@}
\makeatother

\author{
  McClintock, Tom\\
  \texttt{tmcclintock@email.arizona.edu}
}
\title{Ryden - Chapter 3 Solutions}

\begin{document}
\maketitle

%%%%%%%%%%%%%%%%%%%%%%%%%%%%%%%%%%%%
\section{Exercise 1}
\label{sec:c2ex1}

The Universe is electrically neutral on large scales because particle physics requires conservation of charge. Additionally, we see no large scale motion of objects being driven by electromagnetic radiation, suggesting that on large scales everything is neutral.

%%%%%%%%%%%%%%%%%%%%%%%%%%%%%%%%%%%%
\section{Exercise 2}
\label{sec:c2ex2}







\end{document}
