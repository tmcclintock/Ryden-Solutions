\documentclass[12pt,a4paper]{article}
\usepackage{amsmath}
\usepackage[english]{babel}
\usepackage{graphicx}
\usepackage{listings}
\usepackage{fullpage}
\usepackage[T1]{fontenc}
\usepackage{enumerate}
\usepackage[makeroom]{cancel}
\usepackage{hyperref}
\usepackage{natbib}
\bibliographystyle{apj}
\usepackage[labelfont=bf]{caption}
\usepackage[labelsep=period]{caption}


\lstdefinestyle{custompython}{
  belowcaptionskip=1\baselineskip,
  breaklines=true,
  frame=L,
  xleftmargin=\parindent,
  language=bash,
  basicstyle=\footnotesize\ttfamily,
  showstringspaces=false,
  %commentstyle=\itshape\color{purple!40!black},
  %keywordstyle=\itshape\color{green!40!black},
  %identifierstyle=\color{blue},
  %stringstyle=\color{orange},
}

\usepackage{todonotes}
\newcommand\TM[1]{\todo[color=green!40, inline, size=\small]{TM: #1}}

% Roman numerals command
\makeatletter
\newcommand*{\rom}[1]{\expandafter\@slowromancap\romannumeral #1@}
\makeatother

\author{
  McClintock, Tom\\
  \texttt{tmcclintock@email.arizona.edu}
}
\title{Ryden - Chapter 2 Solutions}

\begin{document}
\maketitle

%%%%%%%%%%%%%%%%%%%%%%%%%%%%%%%%%%%%
\section{Exercise 1}
\label{sec:ex1}

We want to calculate the mean free path $l$ of an arrow fired in a forest, where the trees have radius $R$ and the average number of trees per unit area is $\Sigma$. Thinking about scaling, if the trees were more tightly packed, then $l$ should decrease, so then $l$ should be inversely proportional to $\Sigma$. Similarly, if the trees are larger the arrow will not travel as far, so $l$ should be inversely proportional to $R$. Thus, the solution is $l=(R\Sigma)^{-1}$. Given that $R=1$ m and $\Sigma = 0.005$ m$^{-2}$, this means $l=200$ m. {\it Note} this answer isn't perfect, because the trees protrude forward toward the path of the arrow, but computing this correction is tricky because it involves an integral over the impact parameter $b$ of the arrow with respect to the tree it hits. To a first approximation, the true answer is somewhere between 199 m and 200 m.

%%%%%%%%%%%%%%%%%%%%%%%%%%%%%%%%%%%%
\section{Exercise 2}
\label{sec:ex2}

This problem is identical to Exercise 1 but in three dimensions. The number density of stars is given by $n_*$, and the average stellar radius is $R_*$, meaning the cross-sectional area is $\sigma = 4\pi R_*^2$. The mean free path is again inversely proportional to $n_*$ and $\sigma$, so $l=(\sigma n_*)^{-1}$. Given that $n_* = 10^9$ Mpc$^{-3}$ and $R_* = 7\times 10^8$ m, this means $l = 2.1\times 10^{17}$ Mpc.

Similarly, if galaxies are spaced according to $n_g= 1$ Mpc$^{-3}$ and $R_g=2000$ pc, then the mean free path to hit a galaxy is $l_g = 20000$ Mpc. We see that a photon is fairly likely to pass through a galaxy, but not necessarily hit a star in that galaxy.

%%%%%%%%%%%%%%%%%%%%%%%%%%%%%%%%%%%%
\section{Exercise 3}
\label{sec:ex3}

Let us pretend our bodies are entirely made of water and we absorb all CMB photons that pass through us. Let us further assume that we are a nice, round 100 kg, and also a nice, spherical 1 m radius sphere, so that our volume is $V=4\pi/3$ m$^{3}$ and our surface area is $A=4\pi$ m$^{2}$. We know from the text that the number density of CMB photons is $n_\gamma = 4.11\times 10^{8}$ m$^{-3}$, all of which are traveling at the speed of light $c\approx 3\times 10^8$ m/s. Since we are sitting in a photon bath, the number of photons that hit our spherical body per unit time is given by $N = cn_\gamma A = 1.55\times 10^{18}$ s$^{-1}$. Each of these photons is carrying an energy of $6.34\times10^{-4}$ eV, or about $1.016\times10^{-22}$ J. This means the power delivered by the CMB to our bodies is $P = NE_\gamma = 1.57$ J/s. If our heat capacity is $C = 4200$ J kg$^{-1}$ K$^{-1}$, then we can calculate the temperature change per second as ${\rm d}T/{\rm d}t = P/mC = 3.75\times 10^{-10}$ K s$^{-1}$, meaning we would heat up by 1 nanoKelvin in about $t= 2.67$ s.





\end{document}
